\documentclass[xcolor=dvipsnames]{beamer}

\usepackage[UKenglish]{babel}

\usefonttheme{professionalfonts}
\usepackage{fontspec}
\setromanfont{TeX Gyre Pagella}
\setsansfont{Lato}

\usepackage{mathtools}

\usepackage{unicode-math}
\setmathfont{TeX Gyre Pagella Math}

\usepackage{physics}

\usepackage{pgfpages}
%\setbeameroption{show notes on second screen=right}

\definecolor{UBCblue}{rgb}{0.04706, 0.13725, 0.26667}
\definecolor{UBCgreen}{rgb}{0.04706, 0.26667, 0.13725}

\usecolortheme[named=UBCblue]{structure}
\setbeamertemplate{blocks}[rounded]%[shadow=true]
\setbeamercolor{block title}{bg=UBCblue, fg=white}
\setbeamercolor{block body}{bg=UBCblue!20}
\setbeamercolor{block title example}{bg=UBCgreen, fg=white}
\setbeamercolor{block body example}{bg=UBCgreen!20}

% https://tex.stackexchange.com/a/306662
\makeatletter
\def\beamer@framenotesbegin{% at beginning of slide
     \usebeamercolor[fg]{normal text}
      \gdef\beamer@noteitems{}%
      \gdef\beamer@notes{}%
}
\makeatother

\usetheme{Dresden}
\useoutertheme{miniframes} % Alternatively: miniframes, infolines, split
\useinnertheme{circles}
\usefonttheme{professionalfonts}

\title{The consistency of New Foundations}
\subtitle{An alternative foundation for set theory}
\author{Sky Wilshaw, Ya\"el Dillies}
\institute{University of Cambridge}
\date{10th October 2022}

\begin{document}

\begin{frame}
    \titlepage
\end{frame}

\begin{frame}{Outline}
    \tableofcontents[hideallsubsections]
\end{frame}

\section{What is NF?}

\begin{frame}{Set theory and axioms}
    Set theory is used as a foundation for modern mathematics.

    \medskip

    We need to be careful about what we allow to be a set, otherwise we can get paradoxes.

    \[ R := \qty{x\mid x\not\in x};\quad R \in R \iff R \not\in R \]
\end{frame}
\begin{frame}{Zermelo-Fr\"ankel set theory}
    ZF(C) avoids this paradox by restricting set comprehension to only allow specification of subsets of a set that already exists.

    \[ \qty{x\mid x \not\in x} \text{ is not well-formed};\quad \qty{x\in \mathcal U\mid x \not\in x} \text{ is well-formed} \]

    Russell's paradox no longer works.
\end{frame}

\end{document}
