\section{Many-sorted model theory}
This is loosely based off the perspective on categorical logic offered by Johnstone in Volume 2 of \emph{Sketches of an Elephant}, and takes heavy inspiration from the \emph{Flypitch project}.
\begin{definition}
  A \emph{\( \Sigma \)-language} consists of a map \( \newoperator{Functions} : \prod_{n : \mathbb N} (\Fin n \to \Sigma) \to \Sigma \to \Type_u \) and a map \( \newoperator{Relations} : \prod_{n : \mathbb N} (\Fin n \to \Sigma) \to \Type_v \).
\end{definition}
\begin{definition}
  Let \( \Phi : \Sigma \to \Sigma' \).
  Let \( L \) be a \( \Sigma \)-language and let \( L' \) be a \( \Sigma' \)-language.
  Then a \emph{\( \Phi \)-morphism} of languages \( L \xrightarrow\Phi L' \) consists of a map
  \[ \newoperator{onFunction} : \prod_{n : \mathbb N} \prod_{A : \Fin n \to \Sigma} \prod_{B : \Sigma} \newoperator{Functions}_L(n, A, B) \to \newoperator{Functions}_{L'}(\Phi \circ A, \Phi(B)) \]
  and a map
  \[ \newoperator{onRelation} : \prod_{n : \mathbb N} \prod_{A : \Fin n \to \Sigma} \newoperator{Relations}_L(n, A) \to \newoperator{Functions}_{L'}(\Phi \circ A) \]
  If \( L, L' \) are \( \Sigma \)-languages, we may simply say that a \emph{morphism} of languages \( L \to L' \) is a \( \id_\Sigma \)-morphism \( L \xrightarrow{\id_\Sigma} L' \).\footnote{It is crucial that \( \id \circ f \equiv f \circ \id \equiv f \) definitionally.}
\end{definition}
\begin{definition}
  Let \( L \) be a \( \Sigma \)-language, and let \( M : \Sigma \to \Type_w \).
  An \emph{\( L \)-structure} on \( M \) consists of a map
  \[ \newoperator{funMap} : \prod_{n : \mathbb N} \prod_{A : \Fin n \to \Sigma} \prod_{B : \Sigma} \newoperator{Functions}(n, A, B) \to \left(\prod_{i : \Fin n} M(A(i)) \right) \to M(B) \]
  and a map
  \[ \newoperator{relMap} : \prod_{n : \mathbb N} \prod_{A : \Fin n \to \Sigma} \newoperator{Relations}(n, A) \to \left(\prod_{i : \Fin n} M(A(i)) \right) \to \Prop \]
  We will write \( f^M \) for \( \newoperator{funMap}(n, A, B, f) \), and similarly \( R^M \) for \( \newoperator{relMap}(n, R) \).
\end{definition}
\begin{definition}
  Let \( L \) be a \( \Sigma \)-language.
  A \emph{morphism} of \( L \)-structures \( M \to N \) consists of functions \( h_A : M_A \to N_A \) such that for all \( n : \mathbb N \), \( A : \Fin n \to \Sigma \), and \( B : \Sigma \), all function symbols \( f : \newoperator{Functions}(n, A, B) \), and all \( x : \prod_{i : \Fin n} M(A(i)) \),
  \[ h_B(f^M(x)) = f^N(i \mapsto h_{A(i)}(x(i))) \]
  and for all relation symbols \( R : \newoperator{Relations}(n, A) \),
  \[ R^M(x) \to R^N(i \mapsto h_{A(i)}(x(i))) \]
\end{definition}
\begin{definition}
  Let \( L \) be a \( \Sigma \)-language, and let \( \alpha : \Type_{u'} \) be a type of variables of sort \( S : \alpha \to \Sigma \).
  An \emph{\( L \)-term on \( \alpha : S \) of type \( A \)}, the type of which is denoted \( \Term_{\alpha:S} A \), is either
  \begin{itemize}
    \item a variable, comprised solely of a name \( n : \alpha \) such that \( S(n) = A \), or
    \item an application of a function symbol, comprised of some \( n : \mathbb N \), a map \( B : \Fin n \to \Sigma \), a function symbol \( f : \newoperator{Functions}(n, B, A) \), and terms \( t : \prod_{i : \Fin n} \Term_{\alpha:S} B(i) \).
  \end{itemize}
\end{definition}
\begin{definition}
  Let \( L \) be a \( \Sigma \)-language, and let \( \alpha : \Type_{u'} \) and \( S : \alpha \to \Sigma \).
  We define the type of \emph{\( L \)-bounded formulae on \( \alpha : S \)} with free variables indexed by \( \alpha \), and \( n \) additional free variables of types \( f : \Fin n \to \Sigma \), denoted \( \BForm_{\alpha:S}^f \), by the following constructors.
  \begin{itemize}
    \item \( \newoperator{falsum} : \BForm_{\alpha:S}^f \);
    \item \( \newoperator{equal} : \prod_{A : \Sigma} \Term_{\alpha \oplus \Fin n : S \oplus f} A \to \Term_{\alpha \oplus \Fin n : S \oplus f} A \to \BForm_{\alpha:S}^f \);
    \item \( \newoperator{rel} : \prod_{m : \mathbb N} \prod_{B : \Fin m \to \Sigma} \prod_{R : \newoperator{Relations}(m, B)} \left( \prod_{i : \Fin m} \Term_{\alpha \oplus \Fin n : S \oplus f} B(i) \right) \to \BForm_{\alpha:S}^f \);
    \item \( \newoperator{imp} : \BForm_{\alpha:S}^f \to \BForm_{\alpha:S}^f \to \BForm_{\alpha:S}^f \); and
    \item \( \newoperator{all} : \prod_{A : \Sigma} \BForm_{\alpha:S}^{A :: f} \to \BForm_{\alpha:S}^f \),
  \end{itemize}
  where the syntax \( A :: f \) denotes the cons operation on maps from \( \Fin n \).\footnote{This is implemented in mathlib as \texttt{Fin.cons}.}
  An \emph{\( L \)-formula on \( \alpha : S \)} with free variables indexed by \( \alpha \) is an inhabitant of \( \BForm_{\alpha:S}^{\#[]} \).\footnote{The expression \( \#[] \) is mathlib's syntax for the unique map \( \Fin 0 \to \Sigma \).}
  An \emph{\( L \)-sentence} is an \( L \)-formula with free variables indexed by \( \newoperator{Empty} : f \), where \( f \) is the unique function \( \newoperator{Empty} \to \Sigma \).
  An \emph{\( L \)-theory} is a set of \( L \)-sentences.
\end{definition}

\section{Term models}

\section{Ambiguity}
% \begin{definition}
%   Let \( L \) be an \( \mathbb N \)-language (that is, a language with sorts indexed by \( \mathbb N \)).
%   A \emph{type raising morphism} is a map of languages \( L \to L \)
% \end{definition}
