\documentclass[a4paper]{article}

\usepackage[textwidth=14cm]{geometry}
\usepackage{xfrac}
\usepackage{polyglossia}
\setdefaultlanguage{english}

\usepackage{amsmath,amssymb}
\usepackage{enumitem}
\usepackage{hyperref}

\usepackage{tikz-cd}

\usepackage[nameinlink, capitalize]{cleveref}

\input{macros_common}

\usepackage{amsthm}
\usepackage{etexcmds}
\usepackage{thmtools}
\usepackage{physics}
\usepackage{parskip}

%\input{macros_print}
\declaretheorem[numberwithin=section]{theorem}
\declaretheorem[sibling=theorem]{proposition}
\declaretheorem[sibling=theorem]{corollary}
\declaretheorem[sibling=theorem]{remark}
\declaretheorem[sibling=theorem]{lemma}
\declaretheorem[sibling=theorem]{definition}
\declaretheorem[sibling=theorem]{example}

\title{Con(NF)}
\author{Con(NF) project contributors}

\begin{document}
\maketitle

\noindent This is a discussion on the freedom of action theorem.

\section{Data we have available}
Let \( \alpha \) be a proper type index throughout.
To state the freedom of action theorem, we have the following data.

For all lower levels \( \beta < \alpha \), and all paths \( A \) from \( \beta \) to \( \alpha \),
we have \textit{core tangle data} at \( \beta \); that is, \( \mathrm{All}_A \to \mathrm{Str}_A \), \( \tau_A \), and the actions of the relevant groups.
We will later represent \( \mathrm{All}_A \) by simply \( \mathrm{All}_\beta \), but this is not required for the proof of freedom of action itself.

Further, when \( A \) is a proper (nontrivial) path, we have the \textit{full tangle data}; that is, the embeddings \( j, k \) and the position function \( \iota \) as well as its properties.
At \( \alpha \) itself, we cannot assume that we have the smallness and the orderings in particular, since we have not yet proven the size of \( \tau_\alpha \).

The connection between these data is that for each \( \gamma < \beta \) and a path \( A \) from \( \beta \) to \( \alpha \), we have a derivative map from \( \gamma \) to \( \alpha \) in \( \mathrm{All}_{\gamma:A} \) where \( \gamma:A \) is the path created from \( \gamma \) and the elements of \( A \).
Further, this derivative map commutes with the map from allowable permutations to structural permutations, as is assumed in the tangle data above.

\section{Stating freedom of action}
\subsection{Rough description}
The idea behind freedom of action is that any `reasonable partial description' of an allowable permutation at level \( \alpha \) actually extends to an \( \alpha \)-allowable permutation \( \pi \).
The statement of freedom of action is essentially a description of what `reasonable' should mean.
\begin{itemize}
  \item The unpacked coherence condition, roughly \( \qty(\pi_A)_{\delta} (f_{\gamma,\delta} \vdot x) = f_{\gamma,\delta}\qty(\qty(\pi_A)_\delta \vdot x) \), should hold for any proper path \( A \colon \beta \to \alpha \).
  \item The specification should be `small on non-flexible things', where \textit{non-flexible} means roughly that anything that could be constrained by these conditions.
  \item The specification must be either `small or total on flexible things'.
\end{itemize}

\subsection{Support conditions}
Recall the notion of a \textit{support condition} for level \( \alpha \), which is either an atom or a near-litter, together with a path from \( -1 \) to \( \alpha \).
Type-theoretically, this is \( \mathrm{Cond}_\alpha = (\tau_{-1} \oplus \mathcal N) \times \mathrm{Path}(-1,\alpha) \), where \( \mathcal N \) is the type of near-litters.
For a support condition \( (x, A) \), we have that \( \pi_A(x) = x \).
A \textit{binary condition} at level \( \alpha \) is a pair of atoms or a pair of near-litters, together with a path: \( \mathrm{Cond}_\alpha^2 = (\tau_{-1}^2 \oplus \mathcal N^2) \times \mathrm{Path}(-1,\alpha) \).
For a condition \( (x,y,A) \), we say \( \pi_A(x) = y \).

Any \( \alpha \)-structural permutation \( \pi \) gives a set of binary conditions, given by the graph of the permutations.
We will implement a specification of an allowable permutation as a certain set of binary conditions.
Note that there is an injection \( \mathrm{Str}_\alpha \hookrightarrow \mathcal P(\mathrm{Cond}_\alpha^2) \).

\subsection{Maps on support conditions}
For any path \( A \colon \beta \to \alpha \), we have a map \( \mathrm{Cond}_\beta^{(2)} \to \mathrm{Cond}_\alpha^{(2)} \) by concatenating this path into the path component: \( (x,B) \mapsto (x,B:A) \). This map applies both in the unary and the binary case. These maps are also suitably functorial.
Hence, we have \( \mathcal P(\mathrm{Cond}_\alpha^{(2)}) \to \mathcal P(\mathrm{Cond}_\beta^{(2)}) \) given by \( S \mapsto \qty{(x,B) \mid (x,B:A) \in S} \) as an inverse image.

We also have two maps \( \mathrm{dom, range} \colon \mathrm{Cond}_\alpha^2 \to \mathrm{Cond}_\alpha \) as maps from \( (x,y,A) \) to \( (x,A) \) and \( (y,A) \), and hence two maps \( \mathrm{dom, range} \colon \mathcal P(\mathrm{Cond}_\alpha^2) \to \mathcal P(\mathrm{Cond}_\alpha) \) as a forward image.

\subsection{Constraining conditions}
We have a relation \( \prec \) (read \textit{`constrains'}) on \( \mathrm{Cond}_\alpha \), where we say
\begin{itemize}
  \item \( (L, A) \prec (x, A) \) when \( x \in L \) and \( L \) is a litter (an atom is constrained by the litter it belongs to);
  \item \( (N^\circ, A) \prec (N, A) \) when \( N \) is a near-litter and not equal to its corresponding litter \( N^\circ \);
  \item \( (x, A) \prec (N, A) \) for all \( x \in N\,\Delta\, N^\circ \);
  \item \( (y, B:(\gamma<\beta):A) \prec (L, A) \) for all paths \( A \colon \beta \to \alpha \), and \( \gamma,\delta < \beta \), and \( L = f^A_{\gamma,\delta}(x), x \in \tau_{\gamma:A} \), and \( (y,B) \in S_x \), where \( S_x \subseteq \mathrm{Cond}_\gamma \) is the designated support of \( x \).
\end{itemize}

\begin{proposition}
  The relation \( \prec \) is well-founded.
\end{proposition}
\begin{proof}
  By the conditions on orderings, if some constraint \( (x, A) \prec (y,B) \) holds, then \( \iota_A(x) < \iota_B(y) \) in \( \mu \).
\end{proof}

\subsection{Properties on sets of unary conditions}
A constraint on a litter \( (L,A) \) is \textit{flexible} if it is not of the form \( f^A_{\gamma,\delta}(x) \) as above.
These are precisely the elements of \( \mathrm{Cond}_\alpha \) that are not constrained by anything.

We say that \( S \subseteq \mathrm{Cond}_\alpha \) is \textit{support-closed} if whenever \( (f^A_{\gamma,\delta}(x),A) \in S \), we have that \( S_{\gamma:A} \subseteq \mathrm{Cond}_\gamma \) supports \( x \in \tau_{\gamma:A} \).

We say that \( S \) is \textit{local} if
\begin{itemize}
  \item for all litters \( L \) such that \( (L,A) \in S \), we have \( (x,A) \in S \) for all \( a \in L \);
  \item for all litters \( L \) such that \( (L,A) \not\in S \), we have that \( \norm{\qty{a \in L \mid (a,A) \in S}} < \kappa \).
\end{itemize}

We say \( S \) is \textit{non-flex-small} if it includes a small amount (\( < \kappa \)-many) of non-flexible litters.

We say \( S \) is \textit{flex-small} if it contains either a small amount of flexible litters or all flexible litters.

\subsection{Properties on sets of binary conditions}
A set \( \sigma \subseteq \mathrm{Cond}^2_\alpha \) of binary conditions is \textit{one-to-one} if it is one-to-one viewed as a binary relation; that is, \( (x,y,A), (x,y',A) \in \sigma \implies y = y' \), and \( (x,y,A),(x',y,A) \in \sigma \implies x = x' \).

We say \( \sigma \) is \textit{coherent} if its domain \( \mathrm{dom}(\sigma) \) and range \( \mathrm{range}(\sigma) \) are both support-closed, and whenever \( (f^A_{\gamma,\delta}(x),L,A) \in \sigma \), we have that \( (\sigma)_{\gamma:A} \), \( L = f^A_{\gamma,\delta}(\sigma \vdot x) \), and the same holds for \( \sigma^{-1} \).
% TODO: what is the action \sigma \vdot x?

\subsection{Property of freedom of action}
The property of \textit{freedom of action} holds if for all choices the data we considered at the start that satisfies the conditions, the specification extends to an allowable permutation \( \pi \).
\begin{theorem}[freedom of action theorem]
  Suppose that for all \( \beta < \alpha \), we have full tangle data for all \( A \colon \gamma \to \beta \), and suitable derivative maps.
  Then
  \begin{enumerate}
    \item This synthesises, by phase 1, into the context above: core tangle data at all paths and full tangle data at all proper paths.
    \item If the earlier assumed tangle levels satisfy freedom of action, so do the newly-synthesised tangles at type \( \alpha \).
  \end{enumerate}
\end{theorem}

\end{document}
